\documentclass[12pt, a4paper]{article}
\usepackage[utf8]{inputenc}
\usepackage[portuguese]{babel}
\usepackage[T1]{fontenc}
\usepackage{geometry}
\usepackage{titlesec}
\usepackage{enumitem}
\usepackage{hyperref}
\usepackage{graphicx}
\usepackage{float}
\usepackage{xcolor}
\usepackage{tcolorbox}
\usepackage{amssymb}
\usepackage{listings}

% Definição de JavaScript para o pacote listings
\lstdefinelanguage{JavaScript}{
  keywords={break, case, catch, continue, debugger, default, delete, do, else, false, finally, for, function, if, in, instanceof, new, null, return, switch, this, throw, true, try, typeof, var, void, while, with, const, let, await, async},
  morecomment=[l]{//},
  morecomment=[s]{/*}{*/},
  morestring=[b]',
  morestring=[b]",
  ndkeywords={class, export, boolean, throw, implements, import, this},
  keywordstyle=\color{blue}\bfseries,
  ndkeywordstyle=\color{darkgray}\bfseries,
  identifierstyle=\color{black},
  commentstyle=\color{purple}\ttfamily,
  stringstyle=\color{red}\ttfamily,
  sensitive=true
}

\geometry{top=3cm, bottom=2cm, left=3cm, right=2cm}

% Configuração para código
\lstset{
    basicstyle=\ttfamily\small,
    breaklines=true,
    frame=single,
    backgroundcolor=\color{gray!10}
}

\title{\textbf{Projeto Final - M\'odulo RESTART + IA}\\ \large Smart Class Q\&A: Painel de D\'uvidas Inteligente\\ \small (Arquitetura Adaptada para AWS Sandbox)}

\author{
    \textbf{Equipe:} \\
    Francine Luize Da Silva Rosa \\
    Jos\'e Anderson Da Silva Costa \\
    Kaique Lima Torres \\
    Lucas Moreira De Araujo \\
    Luciano Silveira Santos Filho \\
    Samilly Soares Vieira \\
    \vspace{0.5cm} \\ 
    \textbf{Instrutor:} Heberton de Oliveira \\
    \textbf{Turma:} BRSAO207
}

\date{\today}

\begin{document}

\maketitle

\begin{tcolorbox}[colback=yellow!10!white,colframe=orange!75!black,title=\textbf{Adaptações para AWS Sandbox}]
Este documento apresenta a arquitetura adaptada do projeto para funcionar dentro das limitações do ambiente AWS Sandbox do Labs Vocareum. As principais adaptações incluem:
\begin{itemize}
    \item Substituição do AWS Bedrock por \textbf{Amazon Comprehend} (disponível no Sandbox)
    \item Uso obrigatório do \textbf{LabRole} para permissões IAM
    \item Otimização para instâncias \textbf{t3.small} (limite do Sandbox)
    \item Arquitetura serverless com \textbf{Lambda + API Gateway}
\end{itemize}
\end{tcolorbox}
\newpage
\tableofcontents
\newpage

\section{Identificação do Problema}

\subsection{Ideia do Projeto}
\textbf{Smart Class Q\&A:} Um sistema inteligente de filtragem e priorização de dúvidas para aulas online.

\subsection{Descrição do Problema}
Em ambientes de ensino remoto (como Google Meet, Zoom ou Teams), o chat de texto é frequentemente inundado por mensagens de interação social, cumprimentos ou comentários paralelos. Isso gera um "ruído" informacional que faz com que dúvidas técnicas importantes dos alunos passem despercebidas pelo instrutor. O resultado é a frustração do aluno, que se sente ignorado, e a sobrecarga cognitiva do professor, que precisa filtrar mensagens enquanto ensina.

\subsection{Stakeholders}
\begin{itemize}
    \item \textbf{Alunos:} Que desejam ter suas dúvidas respondidas prontamente.
    \item \textbf{Professores/Instrutores:} Que precisam focar no conteúdo sem perder o engajamento da turma.
    \item \textbf{Instituições de Ensino:} Interessadas na qualidade e eficiência das aulas remotas.
\end{itemize}

\subsection{Justificativa}
A implementação de um painel paralelo que utiliza Inteligência Artificial para separar automaticamente "Dúvidas" de "Conversas" garante que nenhuma pergunta técnica seja perdida. Isso democratiza a atenção do professor, reduz a ansiedade dos alunos e otimiza o tempo de aula.

\section{Levantamento de Requisitos}

\subsection{Requisitos Funcionais}
\begin{enumerate}
    \item O sistema deve permitir que alunos enviem mensagens de texto através de uma interface web simples.
    \item O sistema deve processar cada mensagem utilizando \textbf{Amazon Comprehend} para análise de sentimento e classificação do conteúdo.
    \item O sistema deve identificar se a mensagem é uma "Dúvida Técnica" ou "Interação Geral".
    \item O sistema deve exibir para o professor um painel exclusivo contendo apenas as dúvidas filtradas.
    \item O sistema deve permitir que o professor marque uma dúvida como "Respondida".
    \item O sistema deve manter histórico persistente de todas as dúvidas para análise posterior.
    \item O sistema deve identificar dúvidas não respondidas para planejamento de aulas futuras.
\end{enumerate}

\subsection{Requisitos Não Funcionais}
\begin{itemize}
    \item \textbf{Latência:} Processamento em menos de 3 segundos
    \item \textbf{Disponibilidade:} Arquitetura serverless (Lambda) para alta disponibilidade
    \item \textbf{Escalabilidade:} DynamoDB suportando picos simultâneos
    \item \textbf{Usabilidade:} Interface intuitiva e não intrusiva
    \item \textbf{Custo:} Otimizada para free tier do AWS Sandbox
\end{itemize}

\subsection{MVP (Produto Mínimo Viável)}
\begin{itemize}
    \item Interface web estática hospedada no S3
    \item API Gateway + Lambda para processamento
    \item Integração com Amazon Comprehend
    \item Algoritmo de classificação por palavras-chave + sentimento
    \item Armazenamento persistente no DynamoDB
    \item Dashboard do professor com filtros
    \item Funcionalidade de atualizar status
\end{itemize}

\section{Arquitetura Técnica}

\subsection{Componentes da Solução}

\subsubsection{Frontend}
\begin{itemize}
    \item HTML/CSS/JavaScript puro
    \item Hospedado no Amazon S3 com website estático
    \item Comunicação via API Gateway REST
\end{itemize}

\subsubsection{API Gateway}
\begin{itemize}
    \item Endpoint REST para receber mensagens
    \item Endpoint para listar dúvidas
    \item Endpoint para atualizar status
    \item CORS habilitado
\end{itemize}

\subsubsection{AWS Lambda (com LabRole)}
\begin{itemize}
    \item \textbf{ProcessarMensagem:} Recebe, classifica e salva
    \item \textbf{ListarDuvidas:} Consulta e retorna dúvidas
    \item \textbf{AtualizarStatus:} Marca como respondida
    \item Runtime: Node.js 18.x
\end{itemize}

\subsubsection{Amazon Comprehend}
\begin{itemize}
    \item DetectSentiment para análise emocional
    \item DetectKeyPhrases para extração de termos
    \item Idioma: Português (pt)
\end{itemize}

\subsubsection{Amazon DynamoDB}
\begin{itemize}
    \item Tabela: SmartClassMessages
    \item Chave Primária: messageId (String)
    \item Índice Secundário: status-timestamp-index
\end{itemize}

\subsection{Fluxo de Funcionamento}

\textbf{Envio de Mensagem (Aluno):}
\begin{enumerate}
    \item Aluno digita mensagem no chat web
    \item Frontend envia POST para API Gateway
    \item Lambda ProcessarMensagem é acionada
    \item Lambda chama Comprehend para análise
    \item Aplica algoritmo de classificação
    \item Salva no DynamoDB com status inicial
    \item Retorna confirmação
\end{enumerate}

\textbf{Visualização (Professor):}
\begin{enumerate}
    \item Professor acessa dashboard
    \item Frontend chama API Gateway (GET /duvidas)
    \item Lambda ListarDuvidas consulta DynamoDB
    \item Retorna apenas mensagens tipo "DÚVIDA"
    \item Frontend exibe com filtros
\end{enumerate}

\section{Algoritmo de Classificação}

O sistema utiliza uma abordagem h\'ibrida, combinando um classificador l\'exico local com an\'alise sem\^antica via Amazon Comprehend:

\begin{lstlisting}[language=JavaScript, caption={Logica de Classificacao Hibrida}]
// 1. Classificacao por Regras (Local)
const classifier = new MessageClassifier();
const classificationResult = classifier.classify(message);

// 2. Validacao com IA (Amazon Comprehend)
const aiValidation = await aiValidator.validate(message, 
                     classificationResult.classification);

// Resultado Final
const finalStatus = aiValidation.classification;
\end{lstlisting}

O classificador local utiliza um sistema de pesos:
\begin{itemize}
    \item \textbf{Termos Técnicos (Peso 10.0):} Presença de palavras como "AWS", "Lambda", "S3", etc.
    \item \textbf{Palavras de Questionamento (Peso 2.5):} "Como", "Porque", "Onde".
    \item \textbf{Sinal de Pontuação (Peso 3.0):} Presença de ponto de interrogação.
    \item \textbf{Threshold:} A mensagem precisa atingir um score mínimo (8.0) e obrigatoriamente conter termos técnicos para ser filtrada como dúvida.
\end{itemize}

\section{Planejamento \'Agil}

\subsection{Organiza\c{c}\~ao em Sprints}

\textbf{Sprint 1 - Infraestrutura (Conclu\'ida):}
\begin{itemize}
    \item Criar tabela DynamoDB
    \item Configurar S3 bucket para frontend
    \item Criar API Gateway com endpoints
    \item Validar LabRole e permiss\~oes
\end{itemize}

\textbf{Sprint 2 - Backend e IA (Conclu\'ida):}
\begin{itemize}
    \item Desenvolver Lambda ProcessarMensagem
    \item Integrar com Amazon Comprehend (Sentiment/KeyPhrases)
    \item Implementar algoritmo de pesos no MessageClassifier
    \item Testes unit\'arios do motor de classifica\c{c}\~ao
\end{itemize}

\textbf{Sprint 3 - Frontend (Conclu\'ida):}
\begin{itemize}
    \item Interface do aluno (Chat responsivo)
    \item Dashboard do professor com cards de status
    \item Integra\c{c}\~ao real com a API Gateway
    \item Alerta sonoro para novas d\'uvidas
\end{itemize}

\textbf{Sprint 4 - Finaliza\c{c}\~ao (Conclu\'ida):}
\begin{itemize}
    \item Implementa\c{c}\~ao de Notifica\c{c}\~oes via SNS (Email Alerta)
    \item Refinamento de UX/UI (Glassmorphism design e Responsividade)
    \item Documenta\c{c}\~ao t\'ecnica completa e Diagramas (XML)
    \item Preparado para o Pitch e Demonstra\c{c}\~ao ao vivo
\end{itemize}

\subsection{Distribui\c{c}\~ao de Tarefas}
\begin{itemize}
    \item \textbf{Sprint 1 (Infraestrutura):} Jos\'e Anderson
    \item \textbf{Sprint 2 (Backend/IA):} Jos\'e Anderson
    \item \textbf{Sprint 3 (Frontend):} Jos\'e Anderson
    \item \textbf{Sprint 4 (Notifica\c{c}\~ao e Refinamento):} Jos\'e Anderson
    \item \textbf{Documenta\c{c}\~ao e Pitch:} Jos\'e Anderson
\end{itemize}

\section{Estrutura do DynamoDB}

A tabela \textbf{SmartClassMessages} utiliza uma estrutura otimizada para consultas de dashboard:

\begin{lstlisting}[caption={Schema da Tabela DynamoDB}]
{
  "messageId": "1734892800000", // Partition Key
  "timestamp": 1734892800000,    // Sort Key (Unix MS)
  "email": "Jose Anderson",      // Identificador do Aluno
  "message": "Como configurar o GSI?",
  "classification": "DUVIDA",
  "confidence": 95.5,
  "status": "PENDING",           // GSI Partition Key
  "ruleClassification": "DUVIDA",
  "aiReason": "IA confirmou contexto tecnico AWS"
}
\end{lstlisting}

\subsection{\'Indices (GSI)}
O sistema utiliza um Global Secondary Index chamado \textbf{status-index} para permitir que o professor filtre mensagens pendentes ou respondidas de forma eficiente, sem a necessidade de um Scan completo na tabela.

\section{Diferenciais Implementados}

Al\'em dos requisitos b\'asicos, o projeto inclui:
\begin{itemize}
    \item \textbf{Filtro de Mensagens Vagas:} Mensagens curtas ou sem termos t\'ecnicos s\~ao rejeitadas com uma orienta\c{c}\~ao pedag\'ogica autom\'atica.
    \item \textbf{Alerta Sonoro:} O painel do professor emite um som de notifica\c{c}\~ao em tempo real ao detectar uma nova d\'uvida classificada com alta confian\c{c}a.
    \item \textbf{Interface Glassmorphism:} Design moderno com efeitos de transpar\^encia (Blur) e alta legibilidade.
    \item \textbf{Fallback de IA:} Caso o Amazon Comprehend atinja limites do Sandbox, o sistema opera via classificador l\'exico local sem interrup\c{c}\~ao.
\end{itemize}

\section{Recursos Visuais}

\subsection{Diagrama de Arquitetura}

\begin{figure}[H]
    \centering
    \includegraphics[width=\textwidth]{arquitetura.png}
    \caption{Fluxo de Dados da Arquitetura}
    \label{fig:arquitetura}
\end{figure}


\newpage
\subsection{Board Kanban}

\begin{figure}[H]
    \centering
    \includegraphics[width=\textwidth]{kanban.png}
    \caption{Visao geral do Kanban do projeto Smart Class Q\&A, destacando a distribuicao de prioridades, equipes responsaveis e o status das tarefas nas quatro colunas principais.}
    \label{fig:kanban}
\end{figure}

A imagem do quadro Kanban, mostrou:

\begin{itemize}
    \item \textbf{Backlog:} 2 tarefas futuras
    \item \textbf{To Do:} 3 tarefas planejadas
    \item \textbf{Doing:} 4 tarefas em andamento
    \item \textbf{Done:} 8 tarefas conclu\'idas
    \item \textbf{Progresso:} 47\% completo
\end{itemize}

\section{Sugest\~oes de Melhorias e B\^onus}

Caso haja tempo h\'abil antes da entrega final, as seguintes melhorias est\~ao mapeadas para implementa\c{c}\~ao:

\begin{itemize}
    \item \textbf{Dashboard de M\'etricas:} Inclus\~ao de um gr\'afico simples no painel do professor mostrando o volume de d\'uvidas vs. intera\c{c}\~oes sociais para an\'alise de engajamento da turma.
    \item \textbf{Busca Avan\c{c}ada:} Implementa\c{c}\~ao de filtro por palavras-chave no frontend do professor para localizar d\'uvidas espec\'ificas rapidamente em aulas longas.
    \item \textbf{Exporta\c{c}\~ao CSV:} Bot\~ao para baixar todas as d\'uvidas respondidas em formato CSV para posterior registro em di\'arios de classe.
    \item \textbf{PWA (Progressive Web App):} Configura\c{c}\~ao de manifest e service worker para permitir a instala\c{c}\~ao do Dashboard do Professor como um aplicativo no desktop ou mobile.
\end{itemize}

\section{Considera\c{c}\~oes Finais}

\subsection{Limita\c{c}\~oes do Sandbox}
\begin{itemize}
    \item \textbf{IAM Roles:} Uso obrigat\'orio do \textbf{LabRole} pr\'e-configurado para todas as fun\c{c}\~oes Lambda e servi\c{c}os.
    \item \textbf{Tempo de Sess\~ao:} Limita\c{c}\~ao de 4 horas que exige automa\c{c}\~ao do deploy via scripts (infrastructure/deploy.sh).
    \item \textbf{IA da AWS:} Amazon Comprehend validado com suporte total a Portugu\^es (pt) para Sentimento e Frases-chave.
    \item \textbf{Rede:} Uso de Function URLs e API Gateway para contornar restri\c{c}\~oes de VPC.
\end{itemize}

\subsection{Alternativas Implementadas}
\begin{itemize}
    \item \textbf{Classifica\c{c}\~ao h\'ibrida:} Garante funcionalidade mesmo se houver lat\^encia ou indisponibilidade moment\^anea da IA.
    \item \textbf{Arquitetura Serverless:} Elimina a necessidade de gerenciar inst\^ancias EC2, reduzindo a complexidade no Sandbox.
    \item \textbf{Frontend no S3:} Hospedagem de baixo custo e alta disponibilidade sem necessidade de servidor web ativo.
\end{itemize}

\subsection{Pr\'oximos Passos (M\'odulo IA)}
\begin{enumerate}
    \item Autentica\c{c}\~ao com Cognito
    \item Dashboard de an\'alise de tend\^encias
    \item Notifica\c{c}\~oes via SNS
    \item Machine Learning customizado
\end{enumerate}

\section{Conclus\~ao}

O projeto Smart Class Q\&A foi adaptado com sucesso para funcionar no AWS Sandbox, mantendo todos os requisitos funcionais essenciais. A solu\c{c}\~ao demonstra conhecimento profundo de servi\c{c}os AWS, arquitetura serverless e desenvolvimento \'agil, oferecendo valor real para o contexto educacional.

\vspace{1cm}

\begin{center}
\textbf{Documentação Completa:}\\
\url{https://github.com/dessima/smartclass-qa}\\[0.5cm]
\end{center}

\end{document}